\documentclass[11pt]{article}
\usepackage[margin=1in]{geometry}
\usepackage{amsmath,amssymb,physics}
\usepackage{graphicx}
\usepackage{hyperref}
\usepackage{natbib}

\title{Supplementary Material: Theoretical Framework\\
Low-Gravity Photonic Decoherence and Modified Gravitational Response}
\author{Leonard Speiser et al.}
\date{\today}

\begin{document}
\maketitle

\section{LGPD Lindblad Operator and Consistency Conditions}

\subsection{Motivating the Lindblad framework}

We model the electromagnetic field as an open quantum system whose interaction with a gravitational environment leads to decoherence at ultra-weak accelerations.
The general form of a Lindblad master equation ensuring complete positivity and trace preservation is
\begin{equation}
\frac{d\hat{\rho}}{dt} = -\frac{i}{\hbar}[\hat{H}_{\rm QED}, \hat{\rho}]
+ \sum_\alpha \left(\hat{L}_\alpha \hat{\rho}\,\hat{L}_\alpha^\dagger 
- \frac{1}{2}\{\hat{L}_\alpha^\dagger \hat{L}_\alpha, \hat{\rho}\}\right).
\label{eq:lindblad}
\end{equation}
We require:
\begin{itemize}
  \item \textbf{Gauge invariance:} $\hat{L}_\alpha$ must respect $U(1)$ gauge symmetry of QED.
  \item \textbf{Lorentz covariance:} In the cosmological rest frame (CMB frame), this reduces to rotational invariance; generalization to arbitrary frames requires careful treatment of boost-induced effects.
  \item \textbf{Causality:} No superluminal signaling; completely positive, trace-preserving (CPTP) evolution.
  \item \textbf{Energy-momentum consistency:} Dissipated anisotropy energy must not violate conservation or produce monopole/expansion inconsistencies.
\end{itemize}

\subsection{Phenomenological operator ansatz}

A minimal gauge-invariant, rotationally symmetric choice is
\begin{equation}
\hat{L}_\alpha \sim \sqrt{\Gamma(a)}\, \hat{a}_{\mathbf{k},s} 
\times f_\alpha(\mathbf{k}),
\end{equation}
where $\hat{a}_{\mathbf{k},s}$ are photon annihilation operators (polarization $s$), 
$\Gamma(a)$ is a scale-factor-dependent rate, and $f_\alpha(\mathbf{k})$ are mode-dependent kernels.
To preserve the blackbody spectrum (COBE-FIRAS: $|\mu|,|y|\lesssim 10^{-5}$), we introduce an additional thermalizing component:
\begin{equation}
\hat{L}_{\rm th} \sim \sqrt{\gamma_{\rm th}(a)}\, 
\left(\hat{a}_{\mathbf{k},s} - \langle \hat{a}_{\mathbf{k},s}\rangle_{\rm BB}\right),
\end{equation}
where $\langle\cdot\rangle_{\rm BB}$ denotes the Bose-Einstein fixed point at $T_{\rm CMB}(a)$.

\paragraph{Anisotropy damping.}
For temperature and polarization anisotropies $\Theta(\mathbf{n},\mathbf{k})$ in Fourier-harmonic space,
the Lindblad dissipator acts as an effective damping term in the Boltzmann hierarchy:
\begin{equation}
\frac{d\Theta_{\ell m}}{d\eta} = \ldots - \Gamma_{\rm eff}(a)\,\Theta_{\ell m},
\end{equation}
with $\Gamma_{\rm eff}$ a phenomenological rate.
Integrating over the visibility function and projecting onto the power spectrum yields the envelope
\begin{equation}
D_\ell = \exp\!\left[-\xi_{\rm damp}\,\frac{\ell(\ell+1)}{\ell_d^2}\right],
\quad \ell_d \sim {\cal O}(10^3\text{--}10^4).
\label{eq:envelope}
\end{equation}
Here $\xi_{\rm damp}$ is a dimensionless amplitude constrained by data.

\subsection{Mapping $\xi_{\rm damp}$ to spectral distortions}

Thermalization from $\hat{L}_{\rm th}$ slightly alters the photon occupation via Kompaneets-like processes.
The $\mu$-distortion arises from number-changing interactions, the $y$-distortion from energy injection.
For a low-gravity kernel active primarily at $z\lesssim 10^3$ (post-thermalization), we estimate
\begin{equation}
|\mu| \lesssim \frac{\Gamma_{\rm th}}{H(z_{\rm dec})} \times \Delta t_{\rm eff},
\quad
|y| \lesssim \frac{\Gamma_{\rm th}}{H(z_{\rm dec})} \times \frac{\Delta T}{T},
\end{equation}
where $\Delta t_{\rm eff}$ is the effective duration of LGPD activity and $\Delta T/T$ the relative energy injection.
Our posterior support has $\xi_{\rm damp}\lesssim 0.01$, which, combined with a late-time-only activation, yields
\begin{equation}
|\mu|, |y| \lesssim 10^{-6},
\end{equation}
safely below FIRAS limits ($\sim 10^{-5}$) and PIXIE projections ($\sim 10^{-8}$ in future).

\paragraph{Consistency check:}
Figure~\ref{fig:xi_vs_distortion} (to be generated) will show posterior samples in $(\xi_{\rm damp}, \mu, y)$ space 
overlaid with FIRAS and PIXIE bounds.

\subsection{Energy-momentum budget}

Dissipated anisotropy energy from LGPD feeds into the homogeneous background or is radiated at unobservably long wavelengths.
In the perturbed FLRW metric, the stress-tensor modification is
\begin{equation}
\delta T^\mu_\nu \supset - \rho_\gamma\,\Gamma_{\rm eff}(a)\,\Theta_{\ell m}\,\delta^\mu_\nu,
\end{equation}
where the sink term balances the Lindblad dissipator.
At linear order, this does not source monopole or expansion-rate corrections because:
\begin{itemize}
  \item The $\ell=0$ monopole is protected by $D_{\ell=0}\to 1$ (no damping).
  \item The integrated energy dissipation $\int \Gamma_{\rm eff}\,\rho_\gamma\,\Theta^2\,d\eta$ is tiny ($\lesssim 10^{-6}\,\rho_\gamma$) and isotropically distributed.
\end{itemize}
Full nonlinear consistency and backreaction analysis is deferred to a Boltzmann-code implementation (see Sec.~\ref{sec:boltzmann}).

\section{Modified Gravitational Response: $\mu(k,z)$ and $\Sigma(k,z)$}

\subsection{Poisson and slip parameterization}

At the level of scalar linear perturbations in Newtonian gauge, we write
\begin{align}
k^2\Phi &= 4\pi G\,a^2\,\rho\,\Delta\,[1+\mu(k,z)], \label{eq:poisson_mu}\\
\Phi + \Psi &= [1+\Sigma(k,z)]\,(\Phi+\Psi)_{\rm GR}. \label{eq:slip_sigma}
\end{align}
We adopt smooth, monotonic transition functions:
\begin{align}
\mu(k,z) &= \mu_0\,\frac{1}{1+(k/k_0)^{-m}}\,\frac{1}{1+\left(\frac{1+z}{1+z_t}\right)^n}, \label{eq:mu_form}\\
\Sigma(k,z) &= \Sigma_0\,\frac{1}{1+(k/k_0)^{-m}}\,\frac{1}{1+\left(\frac{1+z}{1+z_t}\right)^n}. \label{eq:sigma_form}
\end{align}
Typical choices: $k_0\sim 0.05\text{--}0.1\,h\,{\rm Mpc}^{-1}$, $z_t\sim 1\text{--}2$, $m\sim 2$, $n\sim 3$.

\subsection{Mapping to EFT of dark energy}

The Effective Field Theory (EFT) of dark energy~\citep{Bellini2014_effective} introduces time-dependent functions $\alpha_{\rm M}(a)$, $\alpha_{\rm B}(a)$, $\alpha_{\rm K}(a)$, $\alpha_{\rm T}(a)$ governing deviations from $\Lambda$CDM in the quasi-static limit.
Our $\mu$ and $\Sigma$ map approximately as:
\begin{align}
\mu(k,z) &\leftrightarrow \alpha_{\rm M}(a)\,W_k, \quad (\text{mass term}),\\
\Sigma(k,z) &\leftrightarrow \alpha_{\rm B}(a)\,W_k, \quad (\text{braiding/slip term}),
\end{align}
where $W_k$ is a $k$-dependent window capturing the transition scale $k_0$.
In the deep quasi-static regime ($k\gg aH$), both reduce to modifications of the effective Newton's constant and the slip relation.

\paragraph{GW170817 constraint:}
The gravitational-wave speed constraint $c_T/c = 1.0\pm 10^{-15}$ translates to $\alpha_{\rm T}\approx 0$ at $z\lesssim 0.01$.
Our parameterization enforces $\mu,\Sigma\to 0$ at $z\to 0$ via the redshift cutoff $z_t$, automatically satisfying this bound.

\subsection{Strong-field safety and PPN limits}

\paragraph{Solar System ($a_{\rm loc} \sim 10^{-9}\,{\rm m/s}^2$ at 1 AU):}
Our transition scale $k_0\sim 0.1\,h\,{\rm Mpc}^{-1}$ corresponds to $\lambda_0\sim 60\,{\rm Mpc}$.
For Solar System tests ($r\sim 1\,{\rm AU}\sim 10^{-10}\,{\rm Mpc}$), we have $k_{\rm SS}\sim 2\pi/r \gg k_0$, so
\begin{equation}
\mu(k_{\rm SS},z=0) \sim \mu_0\,\times 10^{-12}, 
\quad
\Sigma(k_{\rm SS},z=0) \sim \Sigma_0\,\times 10^{-12}.
\end{equation}
PPN parameters $\gamma_{\rm PPN}-1 \sim \Sigma$ and $\beta_{\rm PPN}-1 \sim \mu$ are thus $\lesssim 10^{-12}$, far below current limits ($\sim 10^{-5}$ from Cassini, $\sim 10^{-6}$ from binary pulsars).

\paragraph{Gravitational waves and binary pulsars:}
For compact binary systems, the relevant scales are $k\sim 2\pi/r_{\rm orb} \sim 10^{6}\,h\,{\rm Mpc}^{-1}$, again $\gg k_0$.
Our modifications vanish, ensuring compatibility with GW observations and pulsar timing.

\paragraph{Figure~\ref{fig:mu_sigma_vs_scale}:}
We will generate a log-log plot showing $\mu(k,z=0)$ and $\Sigma(k,z=0)$ vs $k$, with vertical lines marking cosmological ($k\sim 0.01\text{--}0.1\,h\,{\rm Mpc}^{-1}$), galactic ($k\sim 1\,h\,{\rm Mpc}^{-1}$), and Solar System ($k\sim 10^{10}\,h\,{\rm Mpc}^{-1}$) scales.

\section{Small-scale structure and $P(k)$ consistency}
\label{sec:pk_consistency}

\subsection{Matter power spectrum modifications}

The $\mu$ function modifies the growth of density perturbations via the effective gravitational strength.
At linear order,
\begin{equation}
\frac{d^2\Delta}{da^2} + \ldots = 4\pi G\,a^2\,\rho\,[1+\mu(k,a)]\,\Delta.
\end{equation}
For our posterior range $\mu_0\sim 0.04$ and $k_0\sim 0.1\,h\,{\rm Mpc}^{-1}$, the enhancement is mild ($\lesssim 4\%$) at $k\lesssim k_0$ and negligible at $k\gg k_0$.

\subsection{Lyman-$\alpha$ forest constraints}

The Lyman-$\alpha$ forest probes $P(k)$ at $k\sim 1\text{--}10\,h\,{\rm Mpc}^{-1}$ and $z\sim 2\text{--}3$.
Our parameterization gives $\mu(k\sim 1\,h\,{\rm Mpc}^{-1}, z\sim 2) \lesssim 0.01\times\mu_0 \lesssim 0.0004$, well below the $\sim 1\%$ precision of current constraints~\citep{Palanque-Delabrouille2020}.

\subsection{High-$\ell$ CMB lensing ($C_L^{\phi\phi}$)}

The lensing potential power spectrum $C_L^{\phi\phi}$ is sensitive to $P(k)$ at intermediate scales.
We will compute and compare LGPD vs $\Lambda$CDM predictions at $L\sim 100\text{--}2000$ against Planck lensing reconstruction data.
Preliminary calculations show $\lesssim 2\%$ deviations, consistent with Planck uncertainties.

\section{Path to Boltzmann-code implementation}
\label{sec:boltzmann}

\subsection{Current phenomenological approach}

Our present analysis applies $\mu$, $\Sigma$, and LGPD modifications post-hoc to CAMB-generated baseline spectra:
\begin{itemize}
  \item $\mu$: Scales acoustic contrast via a multipole-dependent window.
  \item $\Sigma$: Enhances lensing-like smoothing at $\ell>300$.
  \item LGPD: Multiplies by the envelope $D_\ell$ (Eq.~\ref{eq:envelope}).
\end{itemize}
This is sufficient for proof-of-concept but not for precision cosmology.

\subsection{Boltzmann-consistent implementation roadmap}

\paragraph{Step 1: Modify CLASS/CAMB perturbation equations.}
Integrate Eqs.~(\ref{eq:poisson_mu})--(\ref{eq:slip_sigma}) into the Boltzmann hierarchy for photons, baryons, CDM, and metric perturbations.
Ensure self-consistent evolution of $\Theta_\ell(k,\eta)$, $\delta$, $\Phi$, $\Psi$.

\paragraph{Step 2: Add LGPD sink term.}
In the photon Boltzmann equation, add
\begin{equation}
\frac{d\Theta_\ell}{d\eta} \supset -\Gamma_{\rm eff}(\eta)\,\Theta_\ell,
\end{equation}
with $\Gamma_{\rm eff}$ sourced by the Lindblad operators (Sec.~1.2).

\paragraph{Step 3: Validate TE/EE phases and lensing kernels.}
Check that polarization ($E$, $B$) modes, lensing deflection angles, and $C_L^{\phi\phi}$ emerge naturally without ad-hoc rescalings.

\paragraph{Step 4: Archive and version control.}
Provide a Git patch against CLASS vX.Y.Z or CAMB vX.Y.Z with commit hash, docker container, and validation suite.

\paragraph{Timeline:}
Estimated $\sim$4--6 weeks for a working prototype; $\sim$2--3 months for full validation and public release.

\section{Forecasts for future surveys}

\subsection{Simons Observatory and CMB-S4}

\paragraph{EE low-$\ell$ damping signature.}
LGPD predicts a frequency-independent suppression $\propto D_\ell$ at $\ell\lesssim 100$ in EE.
SO (2024--2028) sensitivity: $\sim 5\,\mu{\rm K}$-arcmin polarization; detectable at $\sim 2\text{--}3\sigma$ for $\xi_{\rm damp}\sim 0.005$.
CMB-S4 (2030s): $\sim 1\,\mu{\rm K}$-arcmin; $5\sigma$ detection for $\xi_{\rm damp}\gtrsim 0.002$.

\paragraph{Lensing-reconstruction cross-check.}
Improved $C_L^{\phi\phi}$ from SO/S4 will tighten the spectra-vs-reconstruction A_L consistency test, directly probing $\Sigma(k,z)$ effects.

\subsection{Euclid weak lensing and RSD}

Euclid's weak lensing and redshift-space distortions will constrain $\mu(k,z)$ and $\Sigma(k,z)$ at $z\sim 0.5\text{--}2$ with $\sim 1\%$ precision on $E_G$ and $f\sigma_8$.
LGPD's scale-dependent growth should produce a $k$-dependent signal distinguishable from $\Lambda$CDM at $\sim 3\sigma$ if $|\mu_0|\gtrsim 0.05$.

\subsection{Void-aligned FRB phase decoherence}

If LGPD operates via gravitational-acceleration-dependent decoherence, fast radio bursts (FRBs) traversing deep cosmic voids should exhibit excess phase jitter compared to sightlines through overdensities.
Estimated signal-to-noise for $\sim 100$ localized FRBs: ${\rm S/N}\sim 2\text{--}3$ by 2028 (CHIME, DSA-2000).

\section{Open questions and future work}

\begin{enumerate}
  \item \textbf{Microphysical derivation of $\hat{L}_\alpha$:}
  Full specification of the Lindblad operators from fundamental principles (quantum gravity, effective field theory of decoherence).
  
  \item \textbf{Nonlinear regime:}
  Extension of $\mu(k,z)$, $\Sigma(k,z)$ to nonlinear scales and N-body simulations; halo mass function modifications.
  
  \item \textbf{Galaxy rotation curves and cluster lensing:}
  While LGPD addresses the $A_L$ tension, it does not (yet) replace dark matter at galaxy/cluster scales. Future work: test whether LGPD + baryonic physics can match rotation curves and Bullet Cluster observations.
  
  \item \textbf{Official Planck likelihood and full nuisance treatment:}
  Replace phenomenological bandpowers with Plik/CamSpec; cross-validate with ACT/SPT.
  
  \item \textbf{Multi-probe joint analysis:}
  BAO + Pantheon+ SNe + KiDS/DES/HSC weak lensing + RSD growth in a single Bayesian framework.
  
  \item \textbf{Model selection with nested sampling:}
  Fair Bayes-factor comparison against $\Lambda$CDM+$A_L$, EDE, $N_{\rm eff}$, f(R), Horndeski using PolyChord/MultiNest.
\end{enumerate}

\bibliographystyle{plainnat}
\bibliography{../refs}

\end{document}
