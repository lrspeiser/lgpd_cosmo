\label{sec:discussion}

\paragraph{Physical interpretation.}
A nonzero $\Sigma_0$ acts to enhance the smoothing of acoustic peaks in the lensed CMB spectra without necessarily changing the lensing reconstruction amplitude by the same factor; it thus offers a controlled way to test whether the $A_L$ preference in TT/TE/EE arises from modified Weyl potentials in the ultra-weak regime. A nonzero $\mu_0$ modifies growth and can be tightly tested with $f\sigma_8$ and weak lensing. The LGPD envelope parameter $\xi_{\rm damp}$ captures a hypothesized, frequency-independent anisotropy damping tied to low-gravity decoherence; spectral-distortion limits and polarization phases restrict it to be small.

\paragraph{Comparison to alternatives.}
Massive neutrinos, early dark energy (EDE), and systematics/foreground modeling can each shift the inferred lensing amplitude from TT/TE/EE in different ways. Our $(\mu,\Sigma)$ approach isolates gravitational response effects and tests their consistency with reconstruction and growth. Unlike EDE, it does not alter early-time background physics if $z_t$ is low; unlike increased $N_{\rm eff}$ or massive neutrinos, it need not suppress small-scale matter power in conflict with lensing/growth if $(\mu,\Sigma)$ are constrained.

\paragraph{Predictions and future tests.}
Beyond $A_L$, the model predicts (i) a small, frequency-independent suppression pattern in EE at low multipoles set by $\xi_{\rm damp}$, (ii) tiny but correlated spectral deviations along deep-void sightlines, and (iii) excess phase decoherence for compact extragalactic sources (FRBs/quasars) seen through voids. Upcoming CMB and galaxy surveys (SO, CMB-S4, Euclid) will increase sensitivity to these signatures and to scale-dependent growth implied by $\mu(k,z)$.

\paragraph{Conclusions.}
We have presented a falsifiable, GR-respecting phenomenology in which ultra-weak-gravity effects are encoded by $(\mu,\Sigma)$ and a small LGPD envelope. Preliminary tests on synthetic data show no spurious preference for deviations from $\Lambda$CDM. With a Boltzmann-consistent implementation and official Planck/BAO/SNe/growth likelihoods, we will deliver quantitative constraints on $(\mu_0,\Sigma_0,\xi_{\rm damp})$ and assess whether modified gravity and photonic decoherence can alleviate the $A_L$ tension without compromising other observables.
