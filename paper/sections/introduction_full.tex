The cosmic microwave background (CMB) provides our most precise window into the early Universe, with the Planck satellite delivering exquisite measurements of temperature and polarization anisotropies across the sky \citep{Planck2018_params,Planck2018_cosmo}. Within the standard $\Lambda$CDM framework, these observations have established a concordance cosmology with percent-level precision on key parameters. However, several persistent tensions have emerged that challenge this picture, notably the $H_0$ tension between early- and late-time measurements \citep{Riess2022,DiValentino2021_tensions} and the $\sigma_8$ tension in the amplitude of matter clustering \citep{Heymans2021,Abbott2022_DES}.

Among these anomalies, one of the most intriguing is the \emph{CMB lensing amplitude excess}. When fitting the Planck temperature and polarization power spectra, the amplitude of gravitational lensing deflections, parametrized by $A_L$, is measured to be $A_L = 1.180 \pm 0.065$ (68\% CL) \citep{Planck2018_lensing}, approximately $2.8\sigma$ higher than the $\Lambda$CDM expectation of $A_L = 1.00$. This tension persists across different combinations of Planck data (TT, TE, EE) and survives independent likelihood analyses \citep{Motherwell2023,Rosenberg2022}. While a $2.8\sigma$ discrepancy has a $\sim$0.5\% probability of being a statistical fluctuation, its consistency across multiple data splits and analyses warrants serious investigation as a potential signal of new physics.

\paragraph{Previous explanations.}
Several hypotheses have been proposed to explain the $A_L$ anomaly:
\begin{itemize}
\item \textbf{Systematic errors:} Unmodeled foregrounds, especially dust contamination in polarization, could mimic enhanced lensing \citep{Calabrese2017,Rosenberg2022}. However, dedicated studies using different foreground masks and component separation methods find the effect robust \citep{Planck2018_lensing}.
\item \textbf{Modified neutrino physics:} Increasing the effective number of relativistic species $N_{\rm eff}$ or the sum of neutrino masses $\sum m_\nu$ can affect lensing, but these modifications typically worsen other tensions or are excluded by laboratory bounds \citep{Archidiacono2020,Couchot2017}.
\item \textbf{Early dark energy:} A component of dark energy present at recombination can alter the sound horizon and geometric distances in ways that affect apparent lensing \citep{Poulin2019,Smith2020_EDE}. However, early dark energy models face challenges from BAO and Ly$\alpha$ forest constraints \citep{Hill2020,Ivanov2020}.
\item \textbf{Modified gravity:} Deviations from general relativity (GR) on cosmological scales can enhance or suppress gravitational potentials and their evolution, directly impacting CMB lensing \citep{Dossett2014,Zumalacarregui2017_hiclass,Bellini2014_effective}. Previous modified gravity tests using CMB data have found no significant deviations \citep{Planck2018_MG,Alam2021_BOSS}, but most focused on large-scale structure rather than CMB lensing specifically.
\end{itemize}

\paragraph{The LGPD hypothesis.}
In this work, we explore a novel phenomenological framework combining two physical mechanisms: (1) \emph{Low-Gravity Photonic Decoherence} (LGPD), in which photons traversing regions of extremely weak gravitational potential undergo environment-induced decoherence \citep{Bassi2013_models,Adler2007_gravdecoherence}, and (2) scale- and redshift-dependent modifications to the gravitational response functions $\mu(k,z)$ and $\Sigma(k,z)$ \citep{Pogosian2010_parameterizing,Silvestri2013_practical,Bellini2014_effective}. The LGPD mechanism introduces a small thermalization rate $\Gamma(a)$ that preferentially damps CMB anisotropies on scales where gravitational potentials are weakest, while the $\mu$-$\Sigma$ framework captures phenomenological departures from GR in the Poisson and slip relations.

Our motivation for this combined framework stems from recent theoretical work on gravitational decoherence \citep{Blencowe2013,Pikovski2015_universal,Bassi2013_models} and observational hints that gravitational effects may exhibit subtle modifications at low curvature scales \citep{Burrage2018_tests,Koyama2016_cosmotests}. While we do not claim a fundamental derivation from quantum gravity, we aim to parametrize potential effects in a testable way and confront them with data.

\paragraph{Key questions.}
This paper addresses the following questions:
\begin{enumerate}
\item Can phenomenological modifications to gravity and photon propagation reconcile the $A_L$ anomaly with other CMB and large-scale structure observables?
\item What are the tightest constraints on LGPD and modified gravity parameters from current data?
\item Are these modifications consistent with solar system tests, Big Bang nucleosynthesis (BBN), and spectral distortion limits?
\item What predictions do these models make for upcoming surveys (Simons Observatory, CMB-S4, Euclid)?
\end{enumerate}

\paragraph{Main results.}
We perform a comprehensive Bayesian analysis of Planck 2018 TT/TE/EE data combined with baryon acoustic oscillation (BAO) measurements from BOSS \citep{Alam2017_BOSS}, Type Ia supernovae (SNe) from Pantheon \citep{Scolnic2018_Pantheon}, and redshift-space distortion (RSD) growth measurements \citep{Gil-Marin2020_eBOSS}. Our analysis yields the following constraints (68\% CL):
\begin{itemize}
\item LGPD damping parameter: $\xi_{\rm damp} = 0.0036^{+0.0039}_{-0.0026}$
\item Modified Newtonian potential: $\mu_0 = -0.016 \pm 0.20$
\item Gravitational slip: $\Sigma_0 = 0.025^{+0.037}_{-0.031}$
\item Effective lensing amplitude: $A_L^{\rm eff} = 1.005^{+0.013}_{-0.012}$
\end{itemize}
These constraints are consistent with GR ($\mu_0 = \Sigma_0 = 0$) and do not significantly reduce the Planck $A_L$ tension, though they do rule out strong modifications ($|\mu_0|, |\Sigma_0| > 0.3$) at $>$95\% CL. We find $\Delta\chi^2 = -7.7$ relative to $\Lambda$CDM, corresponding to marginal statistical preference (Bayes factor $\sim 1.5$, ``not worth more than a bare mention''). Importantly, our results remain consistent with BBN, solar system tests, and FIRAS spectral distortion limits.

\paragraph{Paper outline.}
The structure of this paper is as follows. In Section~\ref{sec:theory}, we present the theoretical framework for LGPD and the $\mu$-$\Sigma$ parameterization, including their implementation in the Boltzmann hierarchy. Section~\ref{sec:data} describes the observational datasets and likelihood construction. Section~\ref{sec:methods} details our inference methodology, priors, and convergence diagnostics. Section~\ref{sec:results} presents posterior constraints, model comparison statistics, and consistency checks. Section~\ref{sec:robustness} explores systematic uncertainties and robustness tests. Finally, Section~\ref{sec:discussion} discusses the physical interpretation of our findings, implications for cosmological tensions, and prospects for future tests. We conclude in Section~\ref{sec:conclusions}.
