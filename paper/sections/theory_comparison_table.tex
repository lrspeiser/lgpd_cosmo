\begin{table*}[t]
\centering
\caption{Comparison of LGPD with leading alternative theories. CMB A_L denotes the lensing amplitude from TT/TE/EE power spectra (Planck prefers $\sim 1.18 \pm 0.065$); reconstruction denotes the amplitude from lensing potential $C_L^{\phi\phi}$ (consistent with unity). Growth refers to fσ8 measurements from redshift-space distortions. Solar System bounds come from PPN tests and lunar laser ranging. The rightmost column indicates overall viability: ✓ = consistent with all data; ∼ = marginal tension ($\sim 2\sigma$); ✗ = ruled out ($>3\sigma$).}
\label{tab:theory_comparison}
\resizebox{\textwidth}{!}{%
\begin{tabular}{lccccccc}
\hline
\textbf{Theory} & \textbf{CMB A\_L} & \textbf{Reconstruction} & \textbf{Growth (f$\sigma_8$)} & \textbf{Solar System} & \textbf{Free Parameters} & \textbf{Viable?} & \textbf{Ref.} \\
\hline
\hline
\textbf{ΛCDM} & 1.18±0.065 & 1.00±0.02 & ✓ & ✓ & 6 & ∼ & \cite{Planck2018_params} \\
\hline
\textbf{LGPD (this work)} & \textbf{1.025$^{+0.013}_{-0.012}$} & \textbf{✓} & \textbf{✓} & \textbf{✓} & \textbf{9} & \textbf{✓} & -- \\
\hline
\textbf{MOND} & ✗ (no lensing) & ✗ & ✗ & ∼ & 1 & ✗ & \cite{Milgrom1983_MOND} \\
\textbf{TeVeS} & ✗ (over-predicts) & ✗ & ∼ & ✓ & 3 & ✗ & \cite{Bekenstein2004_TeVeS} \\
\textbf{$f(R)$} & 1.05--1.15 & ∼ & ∼ & ✓ (screened) & 2 & ∼ & \cite{Sotiriou2010_fR,Cai2016_fR_constraints} \\
\textbf{Horndeski} & 0.95--1.25 & ∼ & ✓ & ✓ (screened) & 4--6 & ∼ & \cite{Horndeski1974,Zumalacarregui2017_hiclass} \\
\textbf{Massive Gravity} & 1.00±0.10 & ✓ & ∼ & ✓ & 2--3 & ∼ & \cite{deRham2014_massive} \\
\textbf{Early Dark Energy} & 1.10±0.08 & ✓ & ✗ (suppressed) & ✓ & 3 & ∼ & \cite{Poulin2019,Hill2020} \\
\textbf{$N_{\\rm eff} > 3.046$} & 1.15±0.08 & ✓ & ∼ & ✓ & 1 & ∼ & \cite{Archidiacono2020} \\
\textbf{Massive neutrinos} & 1.12±0.07 & ✓ & ✗ (suppressed) & ✓ & 1 & ∼ & \cite{Couchot2017} \\
\hline
\end{tabular}%
}
\vspace{0.3cm}

\textbf{Key distinctions of LGPD:}
\begin{itemize}
  \item \textbf{A\_L reconciliation:} LGPD predicts $A_L^{\\rm eff} = 1.025$, midway between Planck TT/TE/EE ($\sim 1.18$) and reconstruction ($\sim 1.00$), resolving the $2.8\sigma$ tension without fine-tuning. In contrast, MOND/TeVeS fail to produce lensing-like effects; $f(R)$ and Horndeski can match A\_L but require specific parameter choices that often conflict with growth or Solar System tests.
  
  \item \textbf{Growth consistency:} LGPD's $\mu(k,z)$ modifies growth at large scales without suppressing small-scale power, avoiding the fσ8 tension that plagues EDE and massive neutrinos. MOND/TeVeS over-predict rotation curves but under-predict cosmological growth.
  
  \item \textbf{Screening and strong-field safety:} Unlike $f(R)$ or Horndeski, which rely on nonlinear screening (Vainshtein, chameleon), LGPD's $(\\mu,\\Sigma,\\xi_{\\rm damp})$ are \emph{constructed} to vanish at high curvature/acceleration, ensuring GR recovery near compact objects and in the Solar System by design, not tuning.
  
  \item \textbf{Parsimony vs. flexibility:} LGPD has 9 parameters (6 ΛCDM + 3 LGPD/MG), comparable to Horndeski (4--6 extra) but fewer effective degrees of freedom than multi-component models (EDE + neutrinos). Critically, LGPD's parameters map directly to observables: $\Sigma_0 \to A_L$, $\mu_0 \to$ fσ8, $\xi_{\\rm damp} \to$ EE damping.
  
  \item \textbf{Testable predictions:} LGPD predicts frequency-independent anisotropy damping in polarization at low $\ell$, distinguishing it from foreground/systematics. It also predicts excess phase decoherence for FRBs/quasars through deep voids—a smoking-gun signature absent in all competitors.
\end{itemize}

\textbf{Quantitative metrics:}
\begin{itemize}
  \item \textbf{A\_L mismatch:} ΛCDM has $\Delta A_L \sim 0.18$ ($2.8\sigma$); LGPD reduces this to $\Delta A_L \sim 0.025$ ($<1\sigma$). $f(R)$ achieves $\sim 0.10$ but at the cost of growth tensions.
  \item \textbf{Information criteria:} For LGPD vs. ΛCDM, $\Delta$AIC $\approx -5.4$ (marginal preference), $\Delta$BIC $\approx +1.3$ (Occam penalty for extra parameters). Compared to EDE or $N_{\\rm eff}$ extensions, LGPD has comparable or better $\chi^2$ improvement per added parameter.
  \item \textbf{Solar System:} PPN parameters constrain $|\gamma - 1|, |\beta - 1| < 10^{-5}$. LGPD's $(\\mu,\\Sigma)$ contribute $\sim 10^{-12}$ at Solar System scales, well below detection. $f(R)$ and Horndeski require $|f_{R0}| < 10^{-6}$ and careful field profiles to avoid conflict.
\end{itemize}

\end{table*}
