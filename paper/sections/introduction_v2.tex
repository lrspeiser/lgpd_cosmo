\paragraph{Context and motivation.}
The $\Lambda$CDM model provides an economical account of cosmic expansion, structure formation, and the statistical properties of the cosmic microwave background (CMB). Yet a growing body of precision measurements continues to test this framework at the sub-percent level, exposing mild but persistent tensions. Of particular interest is the CMB \emph{lensing amplitude} discrepancy: when the Planck temperature and polarization power spectra are fit with a phenomenological amplitude parameter $A_L$ that rescales the lensing potential power entering the \emph{lensed} CMB spectra, the best-fit value often exceeds the $\Lambda$CDM expectation $A_L=1$ at the $2$--$3\sigma$ level (see e.g.\ \cite{Planck2018_params,Planck2018_lensing} and follow-ups). In contrast, lensing \emph{reconstruction} from CMB trispectra and independent large-scale structure probes generally prefer amplitudes close to unity. Whether this ``$A_L$ tension'' is a statistical fluctuation, a remnant systematic, or a hint of new physics remains unsettled.

\paragraph{Why focus on ultra-weak gravity?}
General Relativity (GR) has passed the most stringent tests in strong-field regimes: the dynamics of compact binaries, event-horizon-scale imaging of black holes, and Solar-System experiments. The anomalies generally arise in \emph{weak} fields over large distances: galaxy rotation curves, cluster lensing, and certain cosmological inferences. This motivates a class of models in which GR is preserved in strong-field environments but departs, potentially non-perturbatively, in ultra-weak fields or at very low accelerations. We consider here two complementary ingredients: (i) a modified linear response of the metric potentials, encoded by functions $\mu(k,z)$ (modifying the Poisson equation) and $\Sigma(k,z)$ (controlling lensing/gravitational slip), and (ii) a \emph{Low-Gravity Photonic Decoherence} (LGPD) mechanism that alters photon coherence in low-acceleration environments but vanishes in the strong-gravity limit.

\paragraph{Conceptual overview.}
The $\mu$--$\Sigma$ parameterization captures a broad family of modified-gravity (MG) and effective-field-theory (EFT) models in the linear regime; it is agnostic about microphysics but enforces a GR limit as $k\!\to\!\infty$ (or in high-curvature regions). LGPD, by contrast, addresses the photon sector: below a critical acceleration scale, we posit that the photon density matrix experiences gravity-induced decoherence governed by a Lindblad operator, with a small emissive/absorptive component that drives the photon field toward a blackbody fixed point. In anisotropies, LGPD manifests as a \emph{frequency-independent} damping envelope that is negligible in strong fields and small even in the ultra-weak limit, in order to respect spectral-distortion limits. Together, $(\mu,\Sigma)$ and LGPD can (i) alter the \emph{lensed} TT/TE/EE spectra via both potential growth/slip and a gentle anisotropy envelope, (ii) leave CMB lensing \emph{reconstruction} approximately untouched (depending on $\Sigma$), and (iii) preserve the high-precision tests of GR in compact systems.

\paragraph{Relation to prior work.}
Scale-dependent growth modifiers, gravitational slip, and EFT-of-dark-energy approaches have been widely studied as agnostic probes of deviations from GR on cosmological scales (see e.g.\ reviews by \cite{Clifton2012_review,Joyce2016_review} and references therein). Phenomenological $A_L$ fits have been used as diagnostics for anomalous lensing in Planck spectra \cite{Planck2018_lensing}. On the photon-sector side, decoherence and open-quantum-systems techniques have a long history; in cosmology they have been used to explore inflationary decoherence and spectral-distortion physics. Our LGPD proposal is distinct in that (a) the decoherence \emph{trigger} is tied to an invariant of the gravitational field (e.g.\ a local acceleration or tidal scalar), (b) the Lindblad operator includes a controlled emissive term to enforce a Planckian fixed point (thereby respecting $\mu$/$y$ bounds), and (c) the observable consequence in anisotropies is an achromatic damping envelope that scales with a low-gravity rate $\Gamma(a)$ but shuts off in strong fields.

\paragraph{Goals and contributions.}
This paper develops a testable \emph{phenomenology} rather than a fully specified UV completion. Our contributions are:
\begin{itemize}
  \item A mathematically explicit LGPD mechanism expressed via a Lindblad master equation and its line-of-sight radiative-transfer limit, including constraints required to respect spectral-distortion bounds.
  \item A $\mu(k,z)$--$\Sigma(k,z)$ parameterization with smooth transitions in scale and redshift that preserves GR in strong fields and reduces to $\Lambda$CDM as $(\mu,\Sigma,\Gamma)\to 0$.
  \item A Boltzmann-consistent pipeline plan that incorporates $(\mu,\Sigma)$ directly in the perturbation hierarchy, with LGPD as a small, frequency-independent envelope, together with Planck TT/TE/EE+lowE+lensing, BAO, SNe, and $f\sigma_8$ growth likelihoods.
  \item Preliminary constraints using synthetic bandpowers (for pipeline shakedown), showing no spurious preference for $(\mu,\Sigma,\Gamma)\neq 0$ and an effective lensing amplitude consistent with unity.
  \item Clear, falsifiable predictions: tiny direction-dependent spectral deviations along deep-void sightlines, a specific EE suppression pattern at large angular scales, and FRB coherence loss for paths through low-acceleration volumes.
\end{itemize}

\paragraph{Roadmap.}
Section~\ref{sec:theory} formalizes LGPD and the $\mu$--$\Sigma$ response and states theoretical consistency conditions. Section~\ref{sec:data} describes the datasets and likelihoods; Section~\ref{sec:methods} details the parameterization, priors, and sampling. Section~\ref{sec:results} presents constraints and model-comparison metrics. Section~\ref{sec:robust} reports robustness and null tests. We conclude with implications and forecasts in Section~\ref{sec:discussion}.
