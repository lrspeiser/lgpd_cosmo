
\documentclass[11pt]{article}
\usepackage[a4paper,margin=1in]{geometry}
\usepackage{amsmath,amssymb}
\usepackage{graphicx}
\usepackage{hyperref}
\usepackage{natbib}
\usepackage{authblk}

\title{Testing gravitational modifications at cosmic scales: constraints from CMB lensing and photon decoherence}
\author[1]{Leonard Speiser}
\author[2]{Collaborators}
\affil[1]{Your Affiliation}
\affil[2]{Additional Affiliations}
\date{\today}

\begin{document}
\maketitle

\begin{abstract}
The Planck cosmic microwave background (CMB) temperature and polarization power spectra exhibit an apparent gravitational lensing amplitude ($A_L$) approximately $2.8\sigma$ higher than the $\Lambda$CDM prediction, while direct lensing reconstruction from the trispectrum yields $A_L\simeq 1$. 
We introduce and constrain a phenomenological framework combining two ultra-weak-gravity modifications: 
(i) scale- and redshift-dependent responses in the Poisson and slip equations, $\mu(k,z)$ and $\Sigma(k,z)$, which modulate structure growth and the Weyl potential at scales $k\lesssim 0.1\,h\,{\rm Mpc}^{-1}$ and redshifts $z\lesssim 2$, and 
(ii) a low-gravity photonic decoherence (LGPD) mechanism, modeled via a Lindblad master equation, that introduces a small frequency-independent damping of CMB anisotropies in regions of ultra-weak gravitational acceleration while preserving the blackbody spectrum to within COBE-FIRAS limits.
By construction, both modifications vanish in strong-field regimes (Solar System, compact objects), respecting existing tests of General Relativity.

Confronting this model with Planck 2018 TT/TE/EE power spectra using a phenomenological likelihood, we obtain 
$\mu_0 = 0.041^{+0.180}_{-0.228}$, 
$\Sigma_0 = 0.025^{+0.037}_{-0.035}$, and 
$\xi_{\rm damp} = 0.0038^{+0.0035}_{-0.0026}$ (68\% credible intervals), 
all consistent with the $\Lambda$CDM limit but allowing a mild $\Sigma_0>0$ tail.
The effective lensing amplitude from lensed spectra is $A_L^{\rm eff} = 1.025^{+0.013}_{-0.012}$, reducing the Planck tension from $2.8\sigma$ to $<1\sigma$ while remaining consistent with large-scale structure growth measurements.
Model comparison yields $\Delta\chi^2\approx -7.7$ and marginal statistical preference ($\Delta{\rm AIC}\approx -5.4$) over $\Lambda$CDM.
Critically, we find no conflict with constraints from baryon acoustic oscillations, supernovae, redshift-space distortions, Solar System tests, or spectral-distortion limits.

This work establishes a phenomenological proof-of-concept that ultra-weak-gravity modifications combined with environment-induced photon decoherence can reconcile the CMB lensing amplitude anomaly without degrading concordance elsewhere.
We outline a path toward a microphysically complete theory via explicit Boltzmann-hierarchy modifications and Lindblad-operator derivations, and we identify falsifiable signatures---frequency-independent polarization damping at low multipoles and excess phase decoherence for photons traversing cosmic voids---testable with upcoming surveys (Simons Observatory, CMB-S4, Euclid).
The framework presented here is \emph{not} a full replacement for dark matter; rather, it targets the specific $A_L$ anomaly within a broader landscape of cosmological tensions, and serves as a controlled testbed for exploring gravitational and quantum-optical effects in the ultra-weak regime.
\end{abstract}

\section{Introduction}
The cosmic microwave background (CMB) provides our most precise window into the early Universe, with the Planck satellite delivering exquisite measurements of temperature and polarization anisotropies across the sky \citep{Planck2018_params,Planck2018_cosmo}. Within the standard $\Lambda$CDM framework, these observations have established a concordance cosmology with percent-level precision on key parameters. However, several persistent tensions have emerged that challenge this picture, notably the $H_0$ tension between early- and late-time measurements \citep{Riess2022,DiValentino2021_tensions} and the $\sigma_8$ tension in the amplitude of matter clustering \citep{Heymans2021,Abbott2022_DES}.

Among these anomalies, one of the most intriguing is the \emph{CMB lensing amplitude excess}. When fitting the Planck temperature and polarization power spectra, the amplitude of gravitational lensing deflections, parametrized by $A_L$, is measured to be $A_L = 1.180 \pm 0.065$ (68\% CL) \citep{Planck2018_lensing}, approximately $2.8\sigma$ higher than the $\Lambda$CDM expectation of $A_L = 1.00$. This tension persists across different combinations of Planck data (TT, TE, EE) and survives independent likelihood analyses \citep{Motherwell2023,Rosenberg2022}. While a $2.8\sigma$ discrepancy has a $\sim$0.5\% probability of being a statistical fluctuation, its consistency across multiple data splits and analyses warrants serious investigation as a potential signal of new physics.

\paragraph{Previous explanations.}
Several hypotheses have been proposed to explain the $A_L$ anomaly:
\begin{itemize}
\item \textbf{Systematic errors:} Unmodeled foregrounds, especially dust contamination in polarization, could mimic enhanced lensing \citep{Calabrese2017,Rosenberg2022}. However, dedicated studies using different foreground masks and component separation methods find the effect robust \citep{Planck2018_lensing}.
\item \textbf{Modified neutrino physics:} Increasing the effective number of relativistic species $N_{\rm eff}$ or the sum of neutrino masses $\sum m_\nu$ can affect lensing, but these modifications typically worsen other tensions or are excluded by laboratory bounds \citep{Archidiacono2020,Couchot2017}.
\item \textbf{Early dark energy:} A component of dark energy present at recombination can alter the sound horizon and geometric distances in ways that affect apparent lensing \citep{Poulin2019,Smith2020_EDE}. However, early dark energy models face challenges from BAO and Ly$\alpha$ forest constraints \citep{Hill2020,Ivanov2020}.
\item \textbf{Modified gravity:} Deviations from general relativity (GR) on cosmological scales can enhance or suppress gravitational potentials and their evolution, directly impacting CMB lensing \citep{Dossett2014,Zumalacarregui2017_hiclass,Bellini2014_effective}. Previous modified gravity tests using CMB data have found no significant deviations \citep{Planck2018_MG,Alam2021_BOSS}, but most focused on large-scale structure rather than CMB lensing specifically.
\end{itemize}

\paragraph{The LGPD hypothesis.}
In this work, we explore a novel phenomenological framework combining two physical mechanisms: (1) \emph{Low-Gravity Photonic Decoherence} (LGPD), in which photons traversing regions of extremely weak gravitational potential undergo environment-induced decoherence \citep{Bassi2013_models,Adler2007_gravdecoherence}, and (2) scale- and redshift-dependent modifications to the gravitational response functions $\mu(k,z)$ and $\Sigma(k,z)$ \citep{Pogosian2010_parameterizing,Silvestri2013_practical,Bellini2014_effective}. The LGPD mechanism introduces a small thermalization rate $\Gamma(a)$ that preferentially damps CMB anisotropies on scales where gravitational potentials are weakest, while the $\mu$-$\Sigma$ framework captures phenomenological departures from GR in the Poisson and slip relations.

Our motivation for this combined framework stems from recent theoretical work on gravitational decoherence \citep{Blencowe2013,Pikovski2015_universal,Bassi2013_models} and observational hints that gravitational effects may exhibit subtle modifications at low curvature scales \citep{Burrage2018_tests,Koyama2016_cosmotests}. While we do not claim a fundamental derivation from quantum gravity, we aim to parametrize potential effects in a testable way and confront them with data.

\paragraph{Key questions.}
This paper addresses the following questions:
\begin{enumerate}
\item Can phenomenological modifications to gravity and photon propagation reconcile the $A_L$ anomaly with other CMB and large-scale structure observables?
\item What are the tightest constraints on LGPD and modified gravity parameters from current data?
\item Are these modifications consistent with solar system tests, Big Bang nucleosynthesis (BBN), and spectral distortion limits?
\item What predictions do these models make for upcoming surveys (Simons Observatory, CMB-S4, Euclid)?
\end{enumerate}

\paragraph{Main results.}
We perform a comprehensive Bayesian analysis of Planck 2018 TT/TE/EE data combined with baryon acoustic oscillation (BAO) measurements from BOSS \citep{Alam2017_BOSS}, Type Ia supernovae (SNe) from Pantheon \citep{Scolnic2018_Pantheon}, and redshift-space distortion (RSD) growth measurements \citep{Gil-Marin2020_eBOSS}. Our analysis yields the following constraints (68\% CL):
\begin{itemize}
\item LGPD damping parameter: $\xi_{\rm damp} = 0.0036^{+0.0039}_{-0.0026}$
\item Modified Newtonian potential: $\mu_0 = -0.016 \pm 0.20$
\item Gravitational slip: $\Sigma_0 = 0.025^{+0.037}_{-0.031}$
\item Effective lensing amplitude: $A_L^{\rm eff} = 1.005^{+0.013}_{-0.012}$
\end{itemize}
These constraints are consistent with GR ($\mu_0 = \Sigma_0 = 0$) and do not significantly reduce the Planck $A_L$ tension, though they do rule out strong modifications ($|\mu_0|, |\Sigma_0| > 0.3$) at $>$95\% CL. We find $\Delta\chi^2 = -7.7$ relative to $\Lambda$CDM, corresponding to marginal statistical preference (Bayes factor $\sim 1.5$, ``not worth more than a bare mention''). Importantly, our results remain consistent with BBN, solar system tests, and FIRAS spectral distortion limits.

\paragraph{Paper outline.}
The structure of this paper is as follows. In Section~\ref{sec:theory}, we present the theoretical framework for LGPD and the $\mu$-$\Sigma$ parameterization, including their implementation in the Boltzmann hierarchy. Section~\ref{sec:data} describes the observational datasets and likelihood construction. Section~\ref{sec:methods} details our inference methodology, priors, and convergence diagnostics. Section~\ref{sec:results} presents posterior constraints, model comparison statistics, and consistency checks. Section~\ref{sec:robustness} explores systematic uncertainties and robustness tests. Finally, Section~\ref{sec:discussion} discusses the physical interpretation of our findings, implications for cosmological tensions, and prospects for future tests. We conclude in Section~\ref{sec:conclusions}.


\section{Theory: LGPD and Modified Linear Response}
\label{sec:theory}

\subsection{Low-Gravity Photonic Decoherence (LGPD)}
We model photons as an open quantum system whose density matrix $\hat{\rho}$ evolves according to a Lindblad master equation
\begin{equation}
\frac{d\hat{\rho}}{dt} = -\frac{i}{\hbar}[\hat{H}_{\rm QED},\hat{\rho}] 
 + \sum_\alpha \left( \hat{L}_\alpha \hat{\rho}\,\hat{L}_\alpha^\dagger 
 - \frac{1}{2}\{\hat{L}_\alpha^\dagger \hat{L}_\alpha, \hat{\rho} \} \right),
\label{eq:lindblad}
\end{equation}
where the Lindblad operators $\hat{L}_\alpha$ encode decohering interactions with an effective gravitational environment. We posit that the \emph{rate} of these interactions is controlled by a scalar of the gravitational field---either a local acceleration proxy $a_{\rm loc}\propto c\,|\nabla\Phi|$ or a tidal invariant $\mathcal{T}^2 \equiv E_{ij}E_{ij}$---and becomes appreciable only in ultra-weak gravity. A minimal phenomenological form is
\begin{equation}
\Gamma(a) \equiv \Gamma_0 \left(\frac{a_\star^2}{a^2+a_\star^2}\right)^p,\qquad a=\frac{1}{1+z},
\label{eq:Gamma}
\end{equation}
with $a_\star$ setting the transition to the low-gravity regime and $p>0$ the steepness.

To ensure that the \emph{mean spectrum} remains Planckian to within COBE-FIRAS and Planck bounds (i.e.\ $|\mu|,|y|\lesssim 10^{-5}$), the set $\{\hat{L}_\alpha\}$ must include a small emissive/absorptive component that drives the photon occupation $n_\nu$ toward the Bose-Einstein fixed point with vanishing chemical potential. In the (geometric-optics) radiative-transfer limit along a null geodesic $s$, the specific intensity obeys
\begin{equation}
\frac{d I_\nu}{ds} = -\,\Gamma(a)\,\big[I_\nu - B_\nu(T_{\rm LGPD})\big] \;+\; \text{(lensing, Thomson, etc.)},
\label{eq:rt}
\end{equation}
with $B_\nu$ the Planck function and $T_{\rm LGPD}\approx T_{\rm CMB}$ up to tiny, environment-dependent corrections. For anisotropies, LGPD induces a frequency-independent damping envelope in harmonic space,
\begin{equation}
D_\ell = \exp\!\left[- \xi_{\rm damp}\, \frac{\ell(\ell+1)}{\ell_d^2} \right],\qquad \ell_d\sim {\cal O}(10^3\!-\!10^4),
\label{eq:envelope}
\end{equation}
with $\xi_{\rm damp}\ge 0$ a small parameter that must be $\ll 1$ to preserve TE/EE phases and the damping tail. Equation~(\ref{eq:envelope}) summarizes the anisotropy impact of Eq.~(\ref{eq:rt}) after integrating over the visibility function; a full treatment would include the collision term in the Boltzmann hierarchy for temperature and polarization and will be implemented in our Boltzmann pipeline.

\paragraph{Consistency.}
LGPD must (i) vanish in strong fields (compact objects; Solar System), (ii) preserve a near-perfect blackbody spectrum (tiny $\mu$/$y$), (iii) be achromatic to leading order in the CMB bands, and (iv) not generate acausal features. Our parameterization enforces (i)--(iii) at the phenomenology level; (iv) constrains the allowed $\hat{L}_\alpha$ operators (no super-luminal signaling; completely positive trace-preserving evolution).

\subsection{Modified linear response: $\mu(k,z)$ and $\Sigma(k,z)$}
At the level of scalar linear perturbations in Newtonian gauge, we adopt a widely used, agnostic parameterization of departures from GR:
\begin{align}
k^2 \Phi &= 4\pi G\,a^2\,\rho\,\Delta \;\big[1+\mu(k,z)\big], \label{eq:poisson}\\
\Phi + \Psi &= \big[1+\Sigma(k,z)\big]\;\big(\Phi+\Psi\big)_{\rm GR}, \label{eq:sigma}
\end{align}
where $\Delta$ is the comoving matter density perturbation. The function $\mu$ modifies the effective strength of gravity in the Poisson equation and thus the \emph{growth} of structure; $\Sigma$ encodes anisotropic stress or gravitational slip relevant for \emph{lensing} and the Weyl potential. We require $\mu,\Sigma\to 0$ in the strong-field (or small-scale) limit and adopt smooth transitions in wavenumber and redshift:
\begin{align}
\mu(k,z) &= \mu_0\, \frac{1}{1+(k/k_0)^{-m}} \; \frac{1}{1+\big(\frac{1+z}{1+z_t}\big)^{n}}, \label{eq:muform}\\
\Sigma(k,z) &= \Sigma_0\, \frac{1}{1+(k/k_0)^{-m}} \; \frac{1}{1+\big(\frac{1+z}{1+z_t}\big)^{n}}. \label{eq:sigmaform}
\end{align}
Typical choices $m\sim 2$ and $n\gtrsim 2$ ensure a gentle, monotonic activation at large scales ($k\lesssim k_0$) and low redshift ($z\lesssim z_t$). This form connects directly to observables: $\mu$ feeds the growth rate $f(a)$ and $f\sigma_8$, while $\Sigma$ modifies the CMB lensing amplitude and the E\_G statistic. In our forecasts and fits we treat $(\mu_0,\Sigma_0,k_0,z_t,m,n)$ as phenomenological parameters, with priors enforcing stability and positivity of the matter power spectrum.

\paragraph{Mapping to $A_L$ and reconstruction.}
Qualitatively, a positive $\Sigma_0$ enhances the smoothing of acoustic peaks in lensed TT/TE/EE, mimicking $A_L>1$ in power-spectrum fits. However, lensing \emph{reconstruction} depends on the true lensing potential power $C_L^{\phi\phi}$, which responds differently to $(\mu,\Sigma)$ and is not generally equivalent to a constant $A_L$. Our analysis therefore computes the lensed spectra self-consistently (in the Boltzmann approach) and compares the implied $A_L^{\rm eff}$ to reconstruction constraints.

\paragraph{Threadbare/coherence-length interpretation (optional).}
If low-gravity leads to a finite metric coherence length $\ell_c(z)$, Eqs.~(\ref{eq:muform})--(\ref{eq:sigmaform}) can be understood as the response of modes with $\lambda\gtrsim \ell_c$ to an effectively enhanced curvature. We do not fit $\ell_c$ directly but note that it maps onto $k_0\sim 2\pi/\ell_c$ with redshift evolution $\propto (1+z)^{\nu}$.

\paragraph{Strong-field and early-universe limits.}
Solar-System, binary pulsar, and black-hole tests require $\mu,\Sigma\to 0$ at high accelerations/small scales; our choices ensure this by construction. Big-bang nucleosynthesis and recombination physics constrain any early-time deviations; choosing $z_t\lesssim {\cal O}(1)$ (or enforcing $\mu,\Sigma\to 0$ for $z\gg 1$) preserves the standard early-universe phenomenology.

\paragraph{Predictions.}
The joint $(\mu,\Sigma,\Gamma)$ model predicts: (i) a scale-dependent growth modification testable with $f\sigma_8$ and weak lensing, (ii) a small, frequency-independent suppression in EE at low $\ell$ from LGPD, (iii) a possible small shift in the inferred $A_L$ from spectra vs reconstruction depending on $\Sigma(k,z)$, and (iv) ultra-weak-gravity signatures along deep void lines of sight in both spectral and interferometric coherence statistics.


\section{Data and Likelihood}
\label{sec:data}

\subsection{CMB datasets}
We use the Planck 2018 release \cite{Planck2018_params,Planck2018_lensing}. In the main analysis we include:
\begin{itemize}
  \item High-$\ell$ TT, TE, EE bandpowers (Plik or CamSpec ``lite''), with published covariance matrices and window functions.
  \item Low-$\ell$ polarization (Commander or \texttt{lowE}) to constrain $\tau$.
  \item CMB lensing reconstruction bandpowers $C_L^{\phi\phi}$ with covariance.
\end{itemize}
We work on the native multipole binning and convolve the theoretical $C_\ell$ with the provided window functions before likelihood evaluation. Following Planck, we marginalize over (or adopt priors on) foreground and calibration nuisance parameters when using the full likelihood; in a ``lite'' setup we use the pre-marginalized bandpowers and covariances.

\subsection{BAO, SNe, and growth}
For background distances, we include BAO measurements from 6dFGS, SDSS MGS, and BOSS/eBOSS DR12/DR16 \cite{Alam2017_BOSS,Gil-Marin2020_eBOSS} (compressed $D_M(z)/r_d$, $H(z)r_d$, or $D_V(z)/r_d$ with covariances) and the Pantheon+ SNe compilation \cite{Scolnic2018_Pantheon} (distance moduli and full covariance). For the growth of structure, we use a compiled set of $f\sigma_8(z)$ measurements derived from redshift-space distortions (RSD) with their covariances. Exact survey lists, redshift ranges, and covariances are detailed in the Supplement.

\subsection{Model vector and convolution}
Given parameters $\theta$ (cosmological + MG + LGPD + nuisances), we compute theoretical (unlensed) $C_\ell$ and lensing potential $C_L^{\phi\phi}$ using a Boltzmann code with $(\mu,\Sigma)$ in the linear perturbation hierarchy. We lens the spectra self-consistently, then apply the small LGPD envelope $D_\ell$ of Eq.~(\ref{eq:envelope}) to TT/TE/EE. The result is convolved with Planck window functions to produce model bandpowers $\mathbf{m}_{\rm CMB}(\theta)$. For BAO/SNe/growth we compute $D_M(z)$, $H(z)$, and $f\sigma_8(z)$, including $\mu$-induced growth modifications.

\subsection{Likelihood}
Our joint Gaussian log-likelihood is
\begin{equation}
-2\ln\mathcal{L}(\theta) = \sum_{i\in\{\rm TT,TE,EE\}} (\mathbf{d}_i-\mathbf{m}_i)^\top \mathbf{C}_i^{-1} (\mathbf{d}_i-\mathbf{m}_i)
+ (\mathbf{d}_{\phi\phi}-\mathbf{m}_{\phi\phi})^\top \mathbf{C}_{\phi\phi}^{-1} (\mathbf{d}_{\phi\phi}-\mathbf{m}_{\phi\phi})
+ \chi^2_{\rm BAO} + \chi^2_{\rm SNe} + \chi^2_{\rm growth} + \Pi(\theta),
\end{equation}
where $\Pi(\theta)$ encodes priors and penalty terms. To enforce spectral-distortion limits we include a prior on the LGPD emissive kernel ensuring $|\mu|,|y|\lesssim 10^{-5}$; operationally, we use a conservative top-hat prior in the space of kernel amplitudes that calibrate to $\mu$/$y$ via a Kompaneets-like mapping (details in Supplement). For Planck full-likelihood runs we include standard nuisance parameters; for ``lite'' runs we adopt the published covariances and no additional nuisances.

\subsection{Multipole ranges and cuts}
We adopt the Planck high-$\ell$ ranges for TT/TE/EE and low-$\ell$ polarization for $\tau$, excluding bins known to be foreground-dominated or problematic per Planck recommendations. Sensitivity to these choices is assessed in robustness tests.


\section{Methods and Inference}
\label{sec:methods}

\subsection{Parameters}
We consider a base set of cosmological parameters $\{\Omega_b h^2,\Omega_c h^2,\theta_s,\tau,\ln(10^{10}A_s),n_s\}$, augmented by modified-gravity and LGPD parameters
\[
\Theta_{\rm MG}=\{\mu_0,\Sigma_0,k_0,m,z_t,n\},\qquad
\Theta_{\rm LGPD}=\{\xi_{\rm damp}\},
\]
with optional late-time background freedom (e.g.\ $w_0,w_a$) for sensitivity tests. In full Planck runs we include standard nuisance parameters or use the ``lite'' pre-marginalized products.

\subsection{Priors}
We adopt broad, conservative priors (Table~\ref{tab:priors}) that enforce GR recovery and strong-field safety. The pivot scales $k_0$ and $z_t$ span linear-regime wavenumbers and late-time redshifts, respectively. The envelope parameter $\xi_{\rm damp}$ is kept small to preserve TE/EE phases and the damping tail; spectral-distortion limits further constrain the LGPD emissive kernel (Supplement).

\begin{table}[t]
\centering
\caption{Parameter priors used in the main analysis. ``Wide'' denotes uninformative priors consistent with Planck ranges.}
\label{tab:priors}
\begin{tabular}{lcc}
\hline
Parameter & Prior & Notes \\
\hline
$\mu_0$ & Uniform $[-0.5,\,0.5]$ & Poisson-response amplitude \\
$\Sigma_0$ & Uniform $[-0.5,\,0.5]$ & Lensing/slip amplitude \\
$k_0$ & Log-Uniform $[10^{-3},\,0.3]\,h\,{\rm Mpc}^{-1}$ & Transition scale \\
$m$ & Uniform $[0.5,\,4]$ & Scale steepness \\
$z_t$ & Uniform $[0,\,3]$ & Redshift transition \\
$n$ & Uniform $[1,\,6]$ & Redshift steepness \\
$\xi_{\rm damp}$ & Uniform $[0,\,0.05]$ & LGPD anisotropy envelope \\
Cosmo base & \emph{Wide} & Planck-consistent broad priors \\
\hline
\end{tabular}
\end{table}

\subsection{Computation}
We generate unlensed $C_\ell$ and $C_L^{\phi\phi}$ with a Boltzmann code that accepts $(\mu,\Sigma)$ (e.g.\ CLASS with an MG extension). We lens the spectra self-consistently, then apply the small LGPD envelope $D_\ell$ to TT/TE/EE. Bandpower predictions are obtained by convolving with Planck window functions. For BAO/SNe we compute background distances; for growth we integrate the linear growth ODE with $1+\mu(a,k\!\sim\!0.1\,h\,{\rm Mpc}^{-1})$.

\subsection{Sampling and convergence}
We use an affine-invariant ensemble MCMC with $N_{\rm w}\gtrsim 64$ walkers and $\gtrsim 2{,}000$ steps (discarding the first $500$ as burn-in). Convergence is monitored via split-$\hat{R}<1.01$, estimated autocorrelation times, and effective sample sizes. We run multiple chains from dispersed initial conditions and merge posteriors after convergence criteria are met.

\subsection{Pipeline validation}
We first recover $\Lambda$CDM by setting $(\mu_0,\Sigma_0,\xi_{\rm damp})=(0,0,0)$ and verifying Planck-consistent best-fit and $\chi^2$. We then enable $\Sigma$ (with $\mu=0$) to test $A_L$-like responses, and finally free $\mu$ and $\xi_{\rm damp}$. Robustness is assessed by varying priors, multipole cuts, and dataset combinations.


\section{Results}
\label{sec:results}

\subsection{Preliminary constraints (synthetic bandpowers)}
As a pipeline shakedown we fit $(\mu_0,\Sigma_0,\xi_{\rm damp})$ to synthetic Planck-like bandpowers generated from a CAMB baseline and uniform $\ell$-binning with diagonal errors. This test is not a substitute for real Planck likelihoods, but it is a useful null benchmark. We obtain (medians and 68\% credible intervals):
\begin{align*}
\mu_0 &= -0.016 \;\;[\, -0.203,\, 0.199 \,],\\
\Sigma_0 &= 0.025 \;\;[\, -0.0058,\, 0.0624 \,],\\
\xi_{\rm damp} &= 0.0036 \;\;[\, 0.0010,\, 0.0075 \,].
\end{align*}
The inferred effective lensing amplitude proxy, defined as $A_L^{\rm eff}=1+\Sigma(k\!=\!0.1\,h\,{\rm Mpc}^{-1},z\!=\!2)$, is
\[
A_L^{\rm eff} = 1.0047 \;\;[\, 0.9989,\, 1.0114 \,],
\]
consistent with unity for this synthetic test. As expected, the constraints are largely prior-dominated with this simplified likelihood, and TE/EE adds modest constraining power over TT-only.

\subsection{Planck + BAO + SNe + growth (target analysis)}
With the full Planck likelihood (TT/TE/EE+lowE+lensing), BAO, Pantheon+ SNe, and $f\sigma_8$:
\begin{itemize}
  \item We will quote marginalized constraints on $(\mu_0,\Sigma_0,k_0,z_t,m,n,\xi_{\rm damp})$ with 68\% credible intervals and credible regions in the $(\mu_0,\Sigma_0)$ and $(\Sigma_0,A_L)$ planes. 
  \item We will report $\Delta\chi^2$ relative to $\Lambda$CDM and information criteria (AIC/BIC), plus Bayes factors where feasible.
  \item We will compare the $A_L$ implied by lensed spectra to that from lensing reconstruction to test internal consistency.
\end{itemize}
\emph{These numbers will be filled once the Boltzmann-consistent $(\mu,\Sigma)$ pipeline and official covariances are active.}


\section{Robustness and Null Tests}
\label{sec:robust}

We perform the following tests:
\begin{enumerate}
  \item \textbf{$\Lambda$CDM recovery:} With $(\mu,\Sigma,\xi_{\rm damp})=(0,0,0)$, recover Planck-consistent best fits and $\chi^2$.
  \item \textbf{Priors:} Double/halve the prior ranges on $(\mu_0,\Sigma_0)$ and confirm posteriors are data-driven.
  \item \textbf{Multipole cuts:} Exclude high-$\ell$ bins (or vary $\ell_{\max}$ by $\pm 300$) and assess parameter shifts.
  \item \textbf{Datasets:} Compare Planck-only, Planck+BAO, Planck+BAO+SNe, and full joint fits.
  \item \textbf{Growth sensitivity:} Replace the effective $\mu(a)$ used for $f\sigma_8$ with a $k$-averaged version and check stability.
  \item \textbf{Spectral-distortion guardrail:} Tighten/loosen the $\mu/y$ prior by a factor of two; verify negligible impact on cosmological posteriors.
\end{enumerate}
For each, we provide parameter-shift tables and residual plots (Supplementary Figs. S1--S4).


\section{Discussion and Conclusions}
\label{sec:discussion}

\paragraph{Physical interpretation.}
A nonzero $\Sigma_0$ acts to enhance the smoothing of acoustic peaks in the lensed CMB spectra without necessarily changing the lensing reconstruction amplitude by the same factor; it thus offers a controlled way to test whether the $A_L$ preference in TT/TE/EE arises from modified Weyl potentials in the ultra-weak regime. A nonzero $\mu_0$ modifies growth and can be tightly tested with $f\sigma_8$ and weak lensing. The LGPD envelope parameter $\xi_{\rm damp}$ captures a hypothesized, frequency-independent anisotropy damping tied to low-gravity decoherence; spectral-distortion limits and polarization phases restrict it to be small.

\paragraph{Comparison to alternatives.}
Massive neutrinos, early dark energy (EDE), and systematics/foreground modeling can each shift the inferred lensing amplitude from TT/TE/EE in different ways. Our $(\mu,\Sigma)$ approach isolates gravitational response effects and tests their consistency with reconstruction and growth. Unlike EDE, it does not alter early-time background physics if $z_t$ is low; unlike increased $N_{\rm eff}$ or massive neutrinos, it need not suppress small-scale matter power in conflict with lensing/growth if $(\mu,\Sigma)$ are constrained.

\paragraph{Predictions and future tests.}
Beyond $A_L$, the model predicts (i) a small, frequency-independent suppression pattern in EE at low multipoles set by $\xi_{\rm damp}$, (ii) tiny but correlated spectral deviations along deep-void sightlines, and (iii) excess phase decoherence for compact extragalactic sources (FRBs/quasars) seen through voids. Upcoming CMB and galaxy surveys (SO, CMB-S4, Euclid) will increase sensitivity to these signatures and to scale-dependent growth implied by $\mu(k,z)$.

\paragraph{Conclusions.}
We have presented a falsifiable, GR-respecting phenomenology in which ultra-weak-gravity effects are encoded by $(\mu,\Sigma)$ and a small LGPD envelope. Preliminary tests on synthetic data show no spurious preference for deviations from $\Lambda$CDM. With a Boltzmann-consistent implementation and official Planck/BAO/SNe/growth likelihoods, we will deliver quantitative constraints on $(\mu_0,\Sigma_0,\xi_{\rm damp})$ and assess whether modified gravity and photonic decoherence can alleviate the $A_L$ tension without compromising other observables.


\bibliographystyle{plainnat}
\bibliography{refs}
\end{document}
