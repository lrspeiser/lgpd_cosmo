
\documentclass[11pt]{article}
\usepackage[a4paper,margin=1in]{geometry}
\usepackage{amsmath,amssymb}
\usepackage{graphicx}
\usepackage{hyperref}
\usepackage{natbib}
\usepackage{authblk}

\title{Testing gravitational modifications at cosmic scales: constraints from CMB lensing and photon decoherence}
\author[1]{Leonard Speiser}
\author[2]{Collaborators}
\affil[1]{Your Affiliation}
\affil[2]{Additional Affiliations}
\date{\today}

\begin{document}
\maketitle

\begin{abstract}
The Planck cosmic microwave background (CMB) power spectra prefer a lensing amplitude parameter exceeding the $\Lambda$CDM prediction (often summarized as $A_L>1$) at the $2$--$3\sigma$ level, while CMB lensing reconstruction and large-scale structure are broadly consistent with $A_L\simeq 1$. We test whether \emph{ultra-weak-gravity} departures from General Relativity (GR)---encoded by scale- and redshift-dependent response functions $\mu(k,z)$ and $\Sigma(k,z)$---together with a \emph{Low-Gravity Photonic Decoherence} (LGPD) mechanism, can reconcile these datasets without spoiling strong-field tests near compact objects. LGPD is modeled via a Lindblad operator that, in low-acceleration environments, drives the photon field to a Planckian fixed point while introducing a small, frequency-independent damping envelope for anisotropies. We confront the model with Planck TT/TE/EE (plus lensing), BAO, SNe, and $f\sigma_8$ growth. In a preliminary fit to synthetic bandpowers we find constraints consistent with $\Lambda$CDM (e.g.\ $A_L^{\rm eff}=1.005^{+0.007}_{-0.006}$, illustrative only), and we outline a Boltzmann-consistent pipeline that will deliver definitive bounds on $(\mu,\Sigma,\Gamma)$. We conclude with falsifiable predictions---void-direction spectral tightness, a frequency-independent EE suppression pattern, and FRB coherence loss across deep voids---and discuss prospects for SO, CMB-S4, and Euclid.
\end{abstract}

\section{Introduction}

\paragraph{Motivation.} GR passes strong-field tests, yet low-acceleration regimes on galactic/cluster scales show anomalies typically attributed to dark matter and dark energy. We explore an alternative in which spacetime ``unravels'' in low gravity, altering photon coherence (LGPD) and linear gravitational response ($\mu,\Sigma$), while maintaining GR in strong fields.


\section{Theory: LGPD and Modified Linear Response}

\subsection{LGPD}
We posit a Lindblad evolution for the photon density matrix with a low-gravity-triggered rate $\Gamma(a)$ driving the specific intensity $I_\nu$ toward a blackbody $B_\nu(T_{\rm LGPD})$. 
A small emissive/absorptive component ensures $\mu$-type distortions vanish, consistent with FIRAS/Planck.

\subsection{Linear response: $\mu(k,z), \Sigma(k,z)$}
Metric perturbations obey a modified Poisson equation with $1+\mu(k,z)$ and gravitational slip parameterized by $\Sigma(k,z)$, recovering GR as $\mu,\Sigma\to 0$ in strong fields. 
We adopt smooth transitions in $k$ and $z$ for both functions.


\section{Data and Likelihood}

\subsection{CMB}
We use Planck 2018 TT/TE/EE bandpowers with covariances and, where indicated, the lensing reconstruction likelihood.

\subsection{BAO, SNe, Growth}
We include BAO distance measurements, SNe distance moduli, and $f\sigma_8$ growth data with their published covariances. 

\subsection{Spectral distortions}
We enforce $|\mu| \lesssim 9\times 10^{-5}$ and $|y|\lesssim 1.5\times 10^{-5}$ via a prior from a simple distortion calculator.


\section{Methods and Inference}

\subsection{Baseline and modifications}
We generate baseline $C_\ell$ with CAMB/CLASS and apply modifications using $\mu(k,z)$ and $\Sigma(k,z)$ within the Boltzmann hierarchy (preferred), or via a calibrated phenomenological modulator (used as a cross-check).

\subsection{Sampling}
We sample $(\mu_0,\Sigma_0,\xi_{\rm damp},\ldots)$ with wide priors using ensemble MCMC and monitor convergence via split-$\hat R$, autocorrelation times, and ESS.

\subsection{Model comparison}
We report $\Delta\chi^2$ vs.\ $\Lambda$CDM and information criteria (AIC/BIC) for nested models.


\section{Results}

\subsection{CMB constraints}
We present posteriors for $(\mu_0,\Sigma_0,\xi_{\rm damp})$ from TT+TE+EE, and consistency with lensing $A_L$.

\subsection{Joint constraints}
With BAO+SNe+growth, we show that background distances remain consistent while growth/lensing mildly prefer (or disfavor) nonzero $\mu,\Sigma$.

\subsection{Derived parameters}
We report $A_L$ and growth index $\gamma$ in the presence of $\mu,\Sigma$.


\section{Robustness and Null Tests}

We test stability against varying priors, excluding high-$\ell$ bins, changing bin widths, and switching between Plik/CamSpec-style bandpowers. 
We also run a $\Lambda$CDM null test to ensure the pipeline reproduces published Planck goodness-of-fit.


\section{Discussion and Conclusions}

We discuss degeneracies with massive neutrinos, recombination physics, and foregrounds, and outline void-direction tests for LGPD. 
We propose forecasts for next-generation surveys and discuss theoretical implications for microphysical models of LGPD.


\bibliographystyle{plainnat}
\bibliography{refs}
\end{document}
