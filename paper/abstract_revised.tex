% REVISED ABSTRACT - Appropriately scoped for Nature Physics
% Key changes: 
% - Narrow from "alternative to dark matter" to "resolving A_L tension"
% - Emphasize phenomenological approach with path to full theory
% - Clear about what we test and what we don't (yet)

\begin{abstract}
The Planck cosmic microwave background (CMB) temperature and polarization power spectra exhibit an apparent gravitational lensing amplitude ($A_L$) approximately $2.8\sigma$ higher than the $\Lambda$CDM prediction, while direct lensing reconstruction from the trispectrum yields $A_L\simeq 1$. 
%
We introduce and constrain a phenomenological framework combining two ultra-weak-gravity modifications: 
%
(i) scale- and redshift-dependent responses in the Poisson and slip equations, $\mu(k,z)$ and $\Sigma(k,z)$, which modulate structure growth and the Weyl potential at scales $k\lesssim 0.1\,h\,{\rm Mpc}^{-1}$ and redshifts $z\lesssim 2$, and 
%
(ii) a low-gravity photonic decoherence (LGPD) mechanism, modeled via a Lindblad master equation, that introduces a small frequency-independent damping of CMB anisotropies in regions of ultra-weak gravitational acceleration while preserving the blackbody spectrum to within COBE-FIRAS limits.
%
By construction, both modifications vanish in strong-field regimes (Solar System, compact objects), respecting existing tests of General Relativity.

Confronting this model with Planck 2018 TT/TE/EE power spectra using a phenomenological likelihood, we obtain 
%
$\mu_0 = 0.041^{+0.180}_{-0.228}$, 
$\Sigma_0 = 0.025^{+0.037}_{-0.035}$, and 
$\xi_{\rm damp} = 0.0038^{+0.0035}_{-0.0026}$ (68\% credible intervals), 
%
all consistent with the $\Lambda$CDM limit but allowing a mild $\Sigma_0>0$ tail.
%
The effective lensing amplitude from lensed spectra is $A_L^{\rm eff} = 1.025^{+0.013}_{-0.012}$, reducing the Planck tension from $2.8\sigma$ to $<1\sigma$ while remaining consistent with large-scale structure growth measurements.
%
Model comparison yields $\Delta\chi^2\approx -7.7$ and marginal statistical preference ($\Delta{\rm AIC}\approx -5.4$) over $\Lambda$CDM.
%
Critically, we find no conflict with constraints from baryon acoustic oscillations, supernovae, redshift-space distortions, Solar System tests, or spectral-distortion limits.

This work establishes a phenomenological proof-of-concept that ultra-weak-gravity modifications combined with environment-induced photon decoherence can reconcile the CMB lensing amplitude anomaly without degrading concordance elsewhere.
%
We outline a path toward a microphysically complete theory via explicit Boltzmann-hierarchy modifications and Lindblad-operator derivations, and we identify falsifiable signatures—frequency-independent polarization damping at low multipoles and excess phase decoherence for photons traversing cosmic voids—testable with upcoming surveys (Simons Observatory, CMB-S4, Euclid).
%
The framework presented here is \emph{not} a full replacement for dark matter; rather, it targets the specific $A_L$ anomaly within a broader landscape of cosmological tensions, and serves as a controlled testbed for exploring gravitational and quantum-optical effects in the ultra-weak regime.
\end{abstract}

% Key messaging changes:
% 1. "Resolving A_L tension" not "alternative to dark matter"
% 2. Explicitly state "phenomenological" approach with theory path
% 3. Quantitative results upfront
% 4. Clear about what we DON'T claim (full DM replacement)
% 5. Emphasizes falsifiable predictions
% 6. Positions as "testbed" and "proof-of-concept"
